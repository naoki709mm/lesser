\documentclass{jsarticle} 

\usepackage{enumerate}

\title{複素関数論 レポート}
\author{1422020156 奥屋 直己}

\usepackage[height=26cm,width=16cm]{geometry}

\begin{document}

\maketitle

\section{次の式の複素共役な式を作りなさい}
	\begin{enumerate}[(1)]
		\item $\frac{i}{2+3i}$\\
			分母分子にに(2-3i)をかけると\\
			\begin{eqnarray*}
				\frac{i}{2+3i}=\frac{i(2-3i)}{(2+3i)(2-3i)}=\frac{2i-3i^2}{4+9i^2}=\frac{3+2i}{4+9}=-\frac{3}{13}-\frac{2}{13}i
			\end{eqnarray*}
			よって、複素共役数は
			\[
				u^*=-\frac{3}{5}+\frac{2}{5}i
			\]
		\item $e^{(2+5i)π}$ \\
			式変形を行うと
			\[
				e^{(2+5i)\pi}=e^{(2\pi+5i\pi)}=e^{2\pi}e^{5i\pi}
			\]
			オイラーの公式より
			\[
				e^{2\pi}e^{5i\pi}=e^{2\pi}(\cos5\pi+i\sin5\pi)=e^{2\pi}(1+0)=e^{2\pi}
			\]
			よって虚部が0になるため、複素共役は
			\[
				u^*=e^{2\pi}
			\]
		\item $(\sqrt{3}-i)^\frac{3}{2}$\\
			$(\sqrt(3)-i)^3$を展開すると
			\[
				(\sqrt{3}-i)^3=(3-2\sqrt{3}i-1)(\sqrt{3}-i)=(2-2\sqrt{3})(\sqrt{3}-i)=2\sqrt{3}-2i-6i-2\sqrt{3}=-8i
			\]
			よって
			\[
				(\sqrt{3}-i)^\frac{3}{2}=\sqrt{-8i}=2\sqrt{2}i^2=-2\sqrt{2}
			\]
			よって虚部が0になったので、複素共役は
			\[
				u^*=-2\sqrt{2}
			\]
			
		\item $\log{(-2+2i)}$ \\
                  $(-2+2i)$の絶対値を求めると
                  \[
                      \sqrt{(-2)^2+(2)^2}=\sqrt{8}=2\sqrt{2}
                  \]
                  複素数の公式より
                  \[
                      \log{(-2+2i)}=\log{|(-2+2i)|}+i\arg{(-2+2i)}=\log{2\sqrt{2}}+i\arg{(-2+2i)}
                  \]
                  よって複素共役は
                  \[
                  u^*=\log{2\sqrt{2}}-i\arg{(-2+2i)}
                  \]
\end{enumerate}
\section{次の式を証明しなさい}
\begin{enumerate}[(1)]
  \item $(\sin{z})^*=\sin{z^*}$
    オイラーの公式より
    \[
      e^{iz}=\cos{z}+i\sin{z}
    \]
    \[
      e^{-iz}=\cos{z}-i\sin{z}
    \]
    これらを連立方程式で$i\sin{z}$について解く
    \begin{eqnarray*}
      e^{iz}-e^{-iz}&=&2i\sin{z} \\
      \sin{z}&=&\frac{e^{iz}-e^{-iz}}{2i} \\
      両辺にiをかけると&\\
      &=&\frac{i(e^{-iz}-e^{iz})}{2} \\
    \end{eqnarray*}
    $z=a+bi$とおき、上記の式に代入すると
    \begin{eqnarray*}
      \sin{(a+bi)}&=&\frac{i(e^{-i(a+bi)}-e^{i(a+bi)})}{2} \\
      &=&\frac{i(e^{(-ai+b)}-e^{(ai-b)})}{2} \\
      &=&\frac{i(e^be^{-ai}-e^{-b}e^{ai})}{2} \\
      &=&\frac{i((\cosh{b}+\sinh{b})(\cos{a}-i\sin{a})-(\cosh{b}-\sinh{b})(\cos{a}+i\sin{a}))}{2} \\
      &=&\frac{i(-2i\sin{a}\cosh{b}+2\sinh{b}\cos{a})}{2} \\
      &=&\sin{a}\cosh{b}+i\sinh{b}\cos{a}
    \end{eqnarray*}
    また、同様に$\sin{z^*}$は
    \begin{eqnarray*}
      \sin{z^*}=\sin{(a-bi)}\\
      &=&\frac{i(e^{-i(a-bi)}-e^{i(a-bi)})}{2} \\
      &=&\frac{i(-2i\sin{a}\cosh{b}-2\sinh{b}\cos{a})}{2}\\
      &=&\sin{a}\cosh{b}-i\sinh{b}\cos{a}\\
    \end{eqnarray*}
    よって
      \[
        (\sin{z})^*=\sin{a}\cosh{b}-i\sinh{b}\cos{a}=\sin{z^*}
      \]
    となる。
    \item ${\rm arccosh}\ z=\log{\{z+(z^2-1)^{\frac{1}{2}}\}}$
      逆関数の定義より
      \[
      w={\rm arccosh}\ z
      \]
      とおくと
      \[
        z=\cosh{w}=\frac{e^w+e^{-w}}{2}
      \]
      よって
      \[
      u=e^w
      \]
      と置けば
      \[
      z=\cosh{w}=\frac{u+\frac{1}{u}}{2}
      \]
      より
      \begin{eqnarray*}
        2z&=&u+\frac{1}{u}\\
        (u+z)^2-z^2+1&=&z^2-1\\
        (z^2-1)は2価より  \\
        u&=&z+(z^2-1)^{\frac{1}{2}} \\
      \end{eqnarray*}
      よって
      \[
      w={\rm arccosh}\ z=\log{\{z+(z^2-1)^{\frac{1}{2}}\}}
      \]
    \end{enumerate}

\section{次の方程式の解をすべて求めなさい。ただし、zは複素数とする}
\begin{enumerate}[(1)]
  \item $z^6=1$
    式変形すると
    \begin{eqnarray*}
      z^6&=&1\\
      z^6-1&=&0\\
      (z^3-1)(z^3+1)&=&0\\
      (z+1)(z^2-z+1)(z-1)(z^2+z+1)&=&0\\
    \end{eqnarray*}
    $(z^2-z+1)=0$の解は
    \begin{eqnarray*}
      z&=&\frac{1\pm\sqrt{1^2-4*1*1}}{2*1}\\
      &=&\frac{1\pm\sqrt{3}i}{2}
    \end{eqnarray*}
    同様に$(z^2+z+1)=0$の解は
    \begin{eqnarray*}
      z&=&\frac{-1\pm\sqrt{(-1)^2-4*1*1}}{2*1}\\
      &=&\frac{-1\pm\sqrt{3}i}{2}
    \end{eqnarray*}
    よって
    \[
    z=\pm1,\frac{1\pm\sqrt{3}i}{2},\frac{-1\pm\sqrt{3}i}{2}
    \]
  \item $e^{2z}=2$
    
\end{enumerate}
\end{document}












