\documentclass{jsarticle}

\title{レポート課題2}

\date{\today}

\author{奥 屋 直 己}

\usepackage{cases}

\usepackage[height=260mm,width=160mm]{geometry}

\usepackage{listings,jlisting}

\lstset{
  basicstyle={\ttfamily},
  identifierstyle={\small},
  commentstyle={\smallitshape},
  keywordstyle={\small\bfseries},
  ndkeywordstyle={\small},
  stringstyle={\small\ttfamily},
  frame={tb},
  breaklines=true,
  columns=[l]{fullflexible},
  numbers=left,
  xrightmargin=0zw,
  xleftmargin=3zw,
  numberstyle={\scriptsize},
  stepnumber=1,
  numbersep=1zw,
  lineskip=-0.5ex
}

\begin{document}

\maketitle

\section{課題1}
作成したガウス消去法を行うプログラムをソースコード\ref{gauss}に示す。
\lstinputlisting[caption=ガウス消去法,label=gauss]{../gauss.c}
今回、行列の入力を容易にするため、あらかじめ入力すべき行列を保存したファイルを用意し、そのファイル読み込む形式とした。
\subsection{実行例}
以下にガウス消去法の実行例をソースコード\ref{jikkou}に示す。
\begin{lstlisting}[caption=実行結果,label=jikkou]
lesser [java] $ cat 1-1.dat
1 1 1 2
2 3 4 4
2 1 1 3
lesser [java] $ ./gauss 1-1.dat
A=
  1.00   1.00   1.00 
  2.00   3.00   4.00 
  2.00   1.00   1.00 

b=
  2.00
  4.00
  3.00

x=
  1.00
  2.00
 -1.00
lesser [java] $ ./gauss 1-2.dat 
A=
  1.00   2.00   1.00   2.00 
 -1.00  -1.00   0.00  -1.00 
 -2.00  -1.00   2.00   0.00 
  2.00   6.00   1.00   4.00 

b=
  0.00
 -1.00
 -6.00
  5.00

x=
  1.00
  2.00
 -1.00
 -2.00
lesser [java] $ 
\end{lstlisting}
\subsection{解説}
まず、main関数ないで作成しておいたデータが格納されたファイルを読み込む。ファイル名は引数でしていする。ファイル内は行列$\bf{A}$と行列$\bf{b}$を複合した値を保存しておく。これはソースコード2の1行目で、cat関数で表示している。ソースコード1の47行目でchar型の変数f$\_$nameに引数で入力したファイル名を代入し、49行目でファイルを開く。ファイルが開けた場合、まず、ファイルの1行目のみ読み込み、何列で構成されているかを数字の数を読み込むことにより判定し、変数x$\_$nに代入する。59行目のrewind関数を使いもう一度1行目から読み込むようにリセットし、64行目からデータを読み込む。fscanf関数でそれぞれ読み込むが、1行目の終端を読み込む時のみ、改行$\verb|\|$nも読み込む。この読み込んだ値は行列$\bf{A}$と$\bf{b}$が混ざっているので、これらを2次元配列a$\_$bに格納する。80行目より、2次元配列a$\_$bを2次元配列aと配列bに分ける。次にガウス消去法の関数を解説する。このプログラムではガウスの消去法は8行目からの関数で行っている。引数として、行数n、答えを格納する配列x、行列${\bf{A}}$を格納する配列a、右辺の行列$\bf{b}$を格納する行列bをとる。まず12行目からガウスの前進消去を行う。1回目のループで1行目以降の1列目を0にするように計算する。2行目が1行目の何倍となるかを計算した値を変数mikに代入する。これを使いmik倍した1行目引く2行目をし、1列目を0にする。これを右辺の行列bにも同じことをする。3行目も同様に行う。2回目のループで2行目以降の2列目を0にするように計算する。これを繰り返すことで、以下のような形にする。
\[
  \left(
    \begin{array}{ccccc}
      a_{11} & a_{12} & a_{13} & \ldots & a_{1n} \\
      0 & a_{22} & a_{23} &\ldots & a_{2n} \\
      0  & 0 & a_{33} & \ldots & a_{3n} \\
      \vdots & \vdots & \vdots & \ddots & \vdots \\
      0 & 0 & 0 & \ldots & a_{nn}
    \end{array}
  \right)
  \left(
   \begin{array}{c}
   x_1 \\
   x_2 \\
   x_3 \\
   \vdots \\
   x_n
   \end{array}
  \right)=
  \left(
   \begin{array}{c}
   b_1 \\
   b_2 \\
   b_3 \\
   \vdots \\
   b_n
   \end{array}
  \right)
\]
次に23行目からの交代代入について説明する。後退代入ではn行目から走査を始める。n行目では$a_{nn}x_n=b_n$となっているので、$x_n=\frac{b_n}{a_{nn}}$が答えとなる。n-1行の時、$a_{n-1 n-1}x_{n-1}=b_{n-1}-a_nn-1x_n$となる。これを関数内ではx[n-1]にb[n-1]を代入し、それからa[n][n-1]*x[n]を引いた値をx[n-1]に代入。最後にx[n-1]をa[n-1][n-1]で割った値をx[n-1]に代入し、求める。これを繰り返すことで行列$\bf{x}$を求める。

\section{LU分解}
LU分解を行うプログラムをソースコード\ref{LU}に示す。
\lstinputlisting[caption=LU分解,label=LU]{../LU.c}

このプログラムもガウス消去法と同様にファイルからデータを読み込んでいる。ただし、今回は行列Aの部分だけとなっている。
\subsection{実行例}
以下に実行例をソースコード\ref{LUj}に示す。
\begin{lstlisting}[caption=LU分解の実行結果,label=LUj]
lesser [java] $ cat 2-1.dat 
1 1 1
2 3 4
2 1 1
lesser [java] $ ./LU 2-1.dat 
A=
  1.00   1.00   1.00 
  2.00   3.00   4.00 
  2.00   1.00   1.00 

L=
  1.00   0.00   0.00 
  2.00   1.00   0.00 
  2.00  -1.00   1.00 

U=
  1.00   1.00   1.00 
  0.00   1.00   2.00 
  0.00   0.00   1.00 
lesser [java] $ cat 2-2.dat 
1 2 1 2
-1 -1 0 -1
-2 -1 2 0
2 6 1 4
lesser [java] $ ./LU 2-2.dat 
A=
  1.00   2.00   1.00   2.00 
 -1.00  -1.00   0.00  -1.00 
 -2.00  -1.00   2.00   0.00 
  2.00   6.00   1.00   4.00 

L=
  1.00   0.00   0.00   0.00 
 -1.00   1.00   0.00   0.00 
 -2.00   3.00   1.00   0.00 
  2.00   2.00  -3.00   1.00 

U=
  1.00   2.00   1.00   2.00 
  0.00   1.00   1.00   1.00 
  0.00   0.00   1.00   1.00 
  0.00   0.00   0.00   1.00 
lesser [java] $ 
\end{lstlisting}
\subsection{解説}
main関数はガウス消去法とほぼ同じとなっている。この問題では配列lとuをそれぞれ初期化する関数と実際にLU分解を行う関数に分けた。7行目のintlz$\_$lu関数では配列lとuをそれぞれ初期化している。引数として、配列の行数nと行列A、L、Uをそれぞれ格納した配列a、l、uとなっている。配列uは配列aをそのまま代入する。配列lは単位行列で初期化する。すなわち、配列の行番号と列番号が等しい時のみ1.0を代入し、それ以外は0.0を代入する。次に26行目でLU分解を行うlu関数を定義する。引数として、配列の行数nと配列l、uを取っている。LU分解の計算方法は3$\times$3の時だと
\[
  \left(
    \begin{array}{ccc}
      a_{11} & a_{12} & a_{13} \\
      a_{21} & a_{22} & a_{23} \\
      a_{31} & a_{23} & a_{33}
     \end{array}
  \right)=
  \left(
  \begin{array}{ccc}
      1 & 0 & 0 \\
      l_{21} & 1 & 0 \\
      l_{31} & l_{23} & 1
     \end{array}
  \right)
  \left(
    \begin{array}{ccc}
      u_{11} & u_{12} & u_{13} \\
      0 & u_{22} & u_{23} \\
      0 & 0 & u_{33}
     \end{array}
  \right)
 \]
 となればいいので、行列Uの1行目は
\[
 u_{11}=a_{11} \\
\]
\[
 u_{12}=a_{12} \\
\]
\[
 u_{13}=a_{13} \\
\]
となる、次にLの1列目を計算する
\[
 l_{21}=\frac{a_{21}}{u_{11}} \\
\]
\[
 l_{31}=\frac{a_{31}}{u_{11}} \\
\]
となる。さらにUの2行目を計算すると
\[
 u_{22}=a_{22}-l_{21}u_{12} \\
\]
\[
 u_{23}=a_{23}-l_{21}u_{13} \\
\]
同様に3行目も行うことで計算できる。ソースコード3の26行目はこれと同様のことを行っている。配列uはすでに配列aで初期化しているため。uの1行目は計算されている。上記の計算では$\frac{a_{21}}{u_{11}}$だが、配列uは配列aで初期化しているので、$u[2][1]/u[1][1]$で行う。この値を変数mikに代入し、これをl[1][0]に代入し、同じ行と列のu[1][0]に0を代入する。このループ内のまま次の行の配列uを求める。これを繰り返しLU分解を行う。
\section{ドゥーリトル法}
作成したプログラムをソースコード\ref{LU−2}に示す。
\lstinputlisting[caption=ドゥーリトル法,label=LU−2]{../LU-2.c}
\subsection{実行結果}
以下に実行例をソースコード\ref{LU-2j}に示す。
\begin{lstlisting}[caption=ドゥーリトル法の実行結果,label=LU-2j]
lesser [java] $ ./LU-2 1-1.dat
A=
  1.00   1.00   1.00 
  2.00   3.00   4.00 
  2.00   1.00   1.00 

b=
  2.00 
  4.00 
  3.00 

x=
  1.00 
  2.00 
 -1.00 
lesser [java] $ ./LU-2 1-2.dat
A=
  1.00   2.00   1.00   2.00 
 -1.00  -1.00   0.00  -1.00 
 -2.00  -1.00   2.00   0.00 
  2.00   6.00   1.00   4.00 

b=
  0.00 
 -1.00 
 -6.00 
  5.00 

x=
  1.00 
  2.00 
 -1.00 
 -2.00 
lesser [java] $ 
\end{lstlisting}
実行結果よりガウス消去法と値が一致していることがわかる。
\subsection{解説}
このプログラムのmain関数はガウス消去法とほぼ同じとなっている。また、10行目のlu関数もLU分解で説明したものとなっている。このプログラムはLU分解したものから答えとなる行列xを求めるものとなっている。LU分解を行なった結果をそれぞれ配列l,uに格納している。これらとガウス消去法で行なった、後退代入を使い、答えを求める。LU分解を行なったあと、以下のように式を書き換えれる。
\[
\bf{Ax}=\bf{b}
\]
\[
\bf{LUx}=\bf{b}
\]
\[
\bf{Ux}=\bf{y}
\]
とおくと、
\[
\bf{Ly}=\bf{b}
\]
となる。これより、$\bf{Ly}=\bf{b}$を使い$\bf{y}$を求め、$\bf{Ux}=\bf{y}$を使い$\bf{x}$を求める。ソースコード3の45行目より、行列yを求める。行列Lの形はガウスの後退代入を行うときの形の逆なのでループする向きを逆にすれば良い。また、58行目でxを求める時は、ガウスの後退代入がそのまま適応できるので、これからxを求めるとこで、解を求めた。
\section{逆行列のプログラム}
行列Aの逆行列は
\[
A^{-1}=L^{-1}U^{-1}
\]
で求めることができる。3$\times$3の場合、$L^{-1}$と$U^{-1}$は以下のように求めることができる。
 \[
 \left(
  \begin{array}{ccc}
      1 & 0 & 0 \\
      l_{21} & 1 & 0 \\
      l_{31} & l_{23} & 1
     \end{array}
  \right)
  \left(
  \begin{array}{ccc}
      p_{11} & 0 & 0 \\
      p_{21} & p_{22} & 0 \\
      p_{31} & p_{23} & p_{33}
     \end{array}
  \right)=
  \left(
  \begin{array}{ccc}
      1 & 0 & 0 \\
      0 & 1 & 0 \\
      0 & 0 & 1
     \end{array}
  \right)
  \]
\[
\left(
    \begin{array}{ccc}
      u_{11} & u_{12} & u_{13} \\
      0 & u_{22} & u_{23} \\
      0 & 0 & u_{33}
     \end{array}
  \right)
  \left(
    \begin{array}{ccc}
      p_{11} & p_{12} & p_{13} \\
      0 & p_{22} & p_{23} \\
      0 & 0  & p_{33}
     \end{array}
  \right)=
  \left(
  \begin{array}{ccc}
      1 & 0 & 0 \\
      0 & 1 & 0 \\
      0 & 0 & 1
     \end{array}
  \right)
  \]
これらはそれぞれ
\lstinputlisting[caption=$L^{-1}$,label=L−2]{../3.c}
\lstinputlisting[caption=$U^{-1}$,label=U−2]{../4.c}
で求めることができる。これより行列Aの逆数をLUを使い求めることができる。
\section{行列式の計算}
行列Aの行列式は以下のようにして求めることができる。
\[
|A|=|L||U|
\]
また、LとUの行列式はそれぞれ
\[
|L|=l_{11}*l_{22}*\ldots*l_{nn}
\]
\[
|U|=u_{11}*u_{22}*\ldots*u_{nn}
\]
あとはこの二つをかけ合わせると行列Aの行列式が求まる。
\end{document}

